% set paragraph spacing
\setlength{\parskip}{1em}

% renew commands

\renewcommand\thepage{\romannumeral\numexpr\value{page}-1\relax}

% rework glossary title
%\renewcommand{\glossarypreamble}{\vspace*{\baselineskip}\vspace*{\parskip}}

% rework glossary title (better)
\renewcommand{\glossarypreamble}{\vspace{-1.4cm}{\hspace{-0.5cm}\rule{\linewidth}{1pt}\vspace*{3ex}}\protect\thispagestyle{phdfancyempty}}

\renewcommand{\arraystretch}{1.2}

% glossary custom style: 
% https://tex.stackexchange.com/questions/327328/glossaries-style-change-add-units
\newglossarystyle{betterglossary}{%
	\setglossarystyle{list}%

	\renewcommand*{\glossentry}[2]{%

		\item[\glsentryitem{##1}\glstarget{##1}{\glossentryname{##1}}]

		\glossentrydesc{##1} ##2
	}%

	\renewcommand*{\glsgroupskip}{}%
}

%\renewcommand because amssymb package has a command called \bold
\renewcommand{\bold}[1]{\textbf{#1}}

% system modifications
 
\setlength{\heavyrulewidth}{0.07em}
\newcommand{\otoprule}{\midrule[\heavyrulewidth]}

\newdateformat{monthyeardate}{\monthname[\THEMONTH] \THEYEAR}
\newcommand{\monthtoday}{\date{\monthyeardate\today}}

% https://tex.stackexchange.com/questions/112932/today-month-as-text
\newdateformat{coverdate}{\monthname[\THEMONTH] \THEYEAR}

% new commands
\newcommand{\matrixbraket}[2]{
	$\begin{pmatrix}
	#1 \\
	#2 \\
	\end{pmatrix}$
}

\newminted{mumax3}{frame=lines,framerule=2pt}

\newcommand{\newquote}[1]{``#1''}

\newcommand{\italic}[1]{\textit{#1}}

\newcommand{\fixheaderbadness}{
	\hbadness=10000 % A parameter that tells TeX at what point to report badness errors (i.e. underfull and overfull error). [number] ranges from 0 to 10000. 
	\vbadness=10000 % A parameter that tells TeX at what point to report badness errors (i.e. underfull and overfull error). [number] ranges from 0 to 10000. 
	\hfuzz=100pt % A parameter that allows hbox's to be overfull by [length] before an overfull error occurs. 
	\pretolerance=10000
}

\newcommand{\newchapter}[2]{
	
	\protect\chapter{#1}
	\label{#2}
	\setlength{\parsep}{\parskip}%
}

\newcommand{\centerItem}[1]{
	
	\begin{center}
		#1
	\end{center}
}

\newcommand{\imagefigurecaption}[4][0.35]{
	
	\begin{figure}[H]
		\centering
		\includegraphics[width=#1\textwidth,height=\textheight,keepaspectratio]{#2}
		\captionsetup{justification=centering}
		\caption{#3}
		\ifthenelse{\equal{#4}{}}{}{\label{#4}}
	\end{figure}	
}

\newcommand{\imagefigure}[3][0.35]{
	
	\imagefigurecaption[#1]{#2}{#3}{}	
}

% https://tex.stackexchange.com/questions/37581/latex-figures-side-by-side
\newcommand{\asideimages}[8]{
	
	\begin{figure}[H]

		% if captions from subfire are empty, remove the marker below the image
		\ifthenelse{\equal{#4}{}}{
			\ifthenelse{\equal{#6}{}}{
				\captionsetup[subfigure]{labelformat=empty}
			}{}
		}{}
	
		\centering
		\subfloat[#4]{
			{\includegraphics[width=#1]{#3}}
		}
		\qquad
		\subfloat[#6]{
			{\includegraphics[width=#2]{#5}}
		}
		\caption{#7}
		\label{#8}
		
	\end{figure}

}

\newenvironment{changemargin}[2]{%
	
	\begin{list}{}{%
			\setlength{\topsep}{3pt}%
			\setlength{\leftmargin}{#1}%
			\setlength{\rightmargin}{#2}%
			\setlength{\listparindent}{\parindent}%
			\setlength{\itemindent}{\parindent}%
			\setlength{\parsep}{\parskip}%
		}%
	\item[]}{\end{list}}
	
	
\newcommand{\centercaption}[1]{	
	\captionsetup{justification=centering}
	\protect\caption{#1}
}

\DeclareCaptionType{equ}[][]

% https://tex.stackexchange.com/questions/39315/how-to-change-the-numbering-for-different-figures
\DeclareFloatingEnvironment[name={Equation}]{eqfigure}

\newcommand{\setequation}[5] {
	
	\begin{equ}[H]
		
		% https://tex.stackexchange.com/questions/316646/renaming-the-label-of-figure-just-in-some-cases
		%		\renewcommand{\figurename}{Equation}
		\begin{changemargin}{#1}{#2}
			\begin{eqfigure}[H]
				#3
				\vspace{-0.5cm}
				\centercaption{#5}
				\label{#4}
				\addequations{#5} % this is for equations listing
			\end{eqfigure}
		\end{changemargin}	
		\vspace{-1.5cm}
	\end{equ}
}

\newcommand{\centerformula}[1]{
	\vspace{-0.8cm}
	\begin{align}
		#1
	\end{align}
}

\newcommand{\tpower}[1]{
	\ensuremath{\times\ 10^{#1}}
}

\newcommand{\super}[1]{\textsuperscript{#1}}

\newcommand{\sub}[1]{\textsubscript{#1}}

% better diameter symbol
\DeclareFontEncoding{LS1}{}{}
\DeclareFontSubstitution{LS1}{stix}{m}{n}
\DeclareRobustCommand{\diameter}{%
	\text{\usefont{LS1}{stixscr}{m}{n}\symbol{"60}}%
}

\global\def\enableTODO{true}

\newcommand{\showtodos}[1] {
	\global\def\enableTODO{#1}
}

\definecolor{amber}{rgb}{1.0, 0.75, 0.0}

\newcommand{\TODO}[1]{
	\def\trueOption{true}
	\ifx\enableTODO\trueOption
		\ifx\printableimages\trueValue
			\todo[inline, shadow=0.8, bordercolor=black, color=white]{\bold{TODO}: #1}
		\else
			\todo[inline, shadow=0.8, bordercolor=black, color=red!50]{\bold{TODO}: #1}
		\fi
	\fi
}

\newcommand{\sticker}[2][]{
	
	\ignorespaces\lowercase{\def\valueTmp{#1}}
	\def\itemWidth{long}
	
	\ifx\printableimages\trueValue
	
		\ifx\valueTmp\itemWidth
			\todo[inline, shadow=0.8, bordercolor=black, color=white]{#2}
		\else
			\todo[inline, shadow=0.8, bordercolor=black, color=white]{\centerline{#2}}
		\fi
		
	\else
	
		\ifx\valueTmp\itemWidth
			\todo[inline, shadow=0.8, bordercolor=black, color=amber!50]{#2}
		\else
			\todo[inline, shadow=0.8,bordercolor=black, color=amber!50]{\centerline{#2}}
		\fi
		
	\fi
}

% https://tex.stackexchange.com/questions/40561/table-with-multiple-lines-in-some-cells
\newcommand{\celltwolines}[2]{%

	\begin{tabular}{@{}c@{}}
		#1 \\ #2
	\end{tabular}
}

%% WATERMARK STUFF
%Customize watermark
\global\def\optionWatermark{true}

\newcommand{\docwatermark}[1]{
	
	\global\def\optionWatermark{#1}
	\newsavebox\mybox
	\savebox\mybox{\tikz[color=gray,opacity=0.3,font=\sffamily]\node{#1};}
	\newwatermark*[allpages,angle=45,scale=10,xpos=-30,ypos=15]{\usebox\mybox}
}

\newcommand{\showwatermark}[2]{
	
	\def\constPreliminary{PRELIMINARY}
	\def\truestr{true}
	\ignorespaces\lowercase{\def\tmp{#1}}\unskip
	\def\valueNull{\par}
	
	\ifx\tmp\valueNull
		\docwatermark{\constPreliminary}
		\global\def\optionWatermark{true}
	\else
		\ifx\tmp\truestr
			\ignorespaces\def\optValue{#2}\unskip
	
			\ifx\optValue\valueNull
				\docwatermark{\constPreliminary}
			\else
				\def\voidValue{}
				\ifx\optValue\voidValue
					\docwatermark{\constPreliminary}
				\else 
					\docwatermark{\uppercase{#2}}
				\fi
			\fi
	
			\global\def\optionWatermark{true}
		\else
			\global\def\optionWatermark{false}
		\fi
	\fi
}

\newcommand{\preliminary}[2]{
	
	\showwatermark{#1}{#2}
}

\newcommand{\showpreliminary}[2]{
	
	\showwatermark{#1}{#2}
}

\newcommand{\gothicletter}[2][black]{
	
	\yinipar{\color{#1}#2}\hspace{-0.5em}
}

\newcommand{\changeTitleDocument}[1]{
	
	\ignorespaces\lowercase{\def\tmp{#1}}\unskip
	
	\ifx\tmp\empty
		% value empty, nothing to do
	\else
		\global\def\THETITLE{#1}
	\end
}

% rework toc, lot and lof title, add line below
\renewcommand{\cftaftertoctitle}{\\\rule{\linewidth}{1pt}\vspace*{3ex}}
\renewcommand{\cftafterlottitle}{\\\rule{\linewidth}{1pt}\vspace*{3ex}}
\renewcommand{\cftafterloftitle}{\\\rule{\linewidth}{1pt}\vspace*{3ex}}

% rework bibliography title
% https://tex.stackexchange.com/questions/70025/titlesec-and-bibliography
% https://latex.org/forum/viewtopic.php?t=28709
\newcommand{\SetTitleBibliography}[1]{
	\renewcommand{\bibname}{#1}
}

\newcommand{\titlebibliography}[1]{
	\addto{\captionsenglish}{\bibname}
}

\patchcmd{\thebibliography}{\chapter*{\bibname}}{\chapter*{\bibname\\\vspace{-0.5cm}\rule{\linewidth}{1pt}\vspace{-0.5cm}}\protect{\thispagestyle{phdfancyempty}}}{}{}

% redefine plain style to empty style
% https://tex.stackexchange.com/questions/65089/im-trying-to-redefine-the-plain-pagestyle-as-empty
\makeatletter
\let\ps@plain\ps@empty
\makeatother

% create new chapter without page reference in TOC, optional, to set an horizontal line below the title
\newcommandtwoopt{\chaptercustompageintoc}[4][showrule][true]{
	
	% https://tex.stackexchange.com/questions/50038/suppress-page-numbers-when-using-addcontentsline
	\ifthenelse{\equal{#1}{showrule}}{
		
		\chapter*{#3\\\vspace{-0.5cm}\rule{\linewidth}{1pt}\vspace{-1.2cm}}
	}{
		
		\chapter*{{#3}\\\vspace{-1.0cm}}
	}
	
	\ifthenelse{\equal{#2}{true}}{
		\cftaddtitleline{toc}{chapter}{#3}{#4}
	}
}

% plain pagestyle
\newcommandtwoopt{\chapterpagenumberintoc}[3][showrule][true] {
	
	\ifthenelse{\equal{#1}{showrule}}{
		
		\chapter*{#3\\\vspace{-0.5cm}\rule{\linewidth}{1pt}\vspace{-1.2cm}}
	}{
		
		\chapter*{{#3}\\\vspace{-1.0cm}}
	}
	
	\ifthenelse{\equal{#2}{true}}{
		\addcontentsline{toc}{chapter}{#3}
	}
}

%% input file for abstract

\global\def\showpageabstract{false}
\global\def\showrule{showrule}
\global\def\showintoc{true}

\newcommandtwoopt{\inputabstract}[4][false][showrule]{
	
	\ignorespaces\lowercase{\def\tmp{#1}}
	\global\def\showpageabstract{\tmp}
	
	\ignorespaces\lowercase{\def\tmprule{#2}}
	\global\def\showrule{\tmprule}
	
	% whether or not to show in TOC
	\ignorespaces\lowercase{\def\tmptoc{#3}}
	\global\def\showintoc{\tmptoc}
	
	\ifthenelse{\equal{\tmp}{false}}{
		
		% abstract without page numbers and without page number in TOC
		\protect\pagestyle{empty}\input{#4}
		\protect\thispagestyle{empty}
	}{
		
		% abstract with page numers and page number in TOC
		\protect\pagestyle{phdfancyempty}\input{#4}
		\protect\thispagestyle{phdfancyempty}
	}
	
	\clearpage
}

\newcommand{\inputabstracttitle}{
	
	\ifthenelse{\equal{\showpageabstract}{false}}{

		\chaptercustompageintoc[\showrule][\showintoc]{Abstract}{}		
	}{
	
		\chapterpagenumberintoc[\showrule][\showintoc]{Abstract}
	}
}

%% end input file for abstract

% environment for justified text
\newenvironment{justified}[1] {
	\setlength\parindent{0pt}
	\setlength{\parsep}{\parskip}%
	#1
}{}

%% START Create a list of equations

% https://tex.stackexchange.com/questions/270611/creating-list-of-equations
\newcommand{\listequationsname}{List of Equations}
\newcommand{\listequationtitle}{\vspace{-0.4cm}\listequationsname\vskip0cm\vspace{-0.59cm}\rule{\linewidth}{1pt}\vspace{1.8cm}\\\protect{\hspace*{0pt}\hfill\fontsize{11}{12}\textbf{Page}\par\protect\thispagestyle{phdfancyempty}\vspace{-1.0cm}}}
\newlistof{addequations}{equ}{\listequationtitle}
\newcommand{\addequations}[1]{%
	\addcontentsline{equ}{addequations}{\protect\numberline{\theeqfigure}#1}\par}
\addtolength{\cftaddequationsnumwidth}{13pt}
\setlength{\cftaddequationsindent}{1.5em}
\setlength{\cftaddequationsnumwidth}{2.3em}

%% END Create a list of equations

%% START List of Publications 

% USE multibib package: https://www.overleaf.com/learn/latex/Multibib
% https://tex.stackexchange.com/questions/170316/nocite-for-single-bibdatasources-with-biblatex-biber
% https://www.privacypies.org/blog/2017/01/25/bib-management.html
\global\def\listofpublicationsname{List of Publications}

%% END List of Publications

\newcommandtwoopt{\KittelEquationExpandedX}[2][Theoretical Kittel equation expanded for a Permalloy thin-film for X-axe][fig:kittel-x]{
	\vspace{-0.5cm}
	\setequation{0cm}{0cm}{
		\small{\[\textit{f} = 28\cdot\sqrt{(B_{DC}+(N_y-N_x)\cdot0.86\cdot10^6\cdot4\pi\cdot10^{-7})\cdot(B_{DC}+(N_z-N_x)\cdot0.86\cdot10^6)\cdot4\pi\cdot10^{-7}}\]}
	}{#2}{#1}
}

%%% Affirmation
\input{utils/declaration}

%%% FancyHDR page style
% https://en.wikibooks.org/wiki/LaTeX/Customizing_Page_Headers_and_Footers
% remove work chapter from header

\fancypagestyle{phdfancy}{%
	
	\pagestyle{fancy}
	\fancyhead{} % Clean headers
	\fancyhead[LE,RO]{\THETITLE}
	\fancyhead[RE,LO]{\leftmark}
	\fancyfoot[C]{\thepage}
	\renewcommand{\footrulewidth}{0.4pt}
	\renewcommand{\headrulewidth}{0.4pt}
	\renewcommand{\chaptermark}[1]{\markboth{\thechapter. {\slshape{##1}}}{}} 
}

\newcommand{\SetHeaderTitle}[1]{
	\global\def\contentHeaderTitle{#1}
}

\fancypagestyle{phdfancyspecialempty}{%
	
	\pagestyle{fancy}
	\fancyhead{} % Clean headers
	\fancyhead[LE,RO]{\thepage}
	\fancyhead[RE,LO]{\contentHeaderTitle}
	\fancyfoot[C]{}
	\renewcommand{\footrulewidth}{0.0pt}
	\renewcommand{\headrulewidth}{0.4pt}
}

\fancypagestyle{phdfancyempty}{%
	
	\pagestyle{fancy}
	\fancyhead{} % Clean headers
	\fancyhead[LE,RO]{}
	\fancyhead[RE,LO]{}
	\fancyfoot[C]{\thepage}
	\renewcommand{\footrulewidth}{0.0pt}
	\renewcommand{\headrulewidth}{0.0pt}
}

\fancypagestyle{phdplainfancy}{%
	
	\pagestyle{fancy}
	\fancyhead{} % Clean headers
	\fancyhead[LE,RO]{}
	\fancyhead[RE,LO]{}
	\fancyfoot[C]{\thepage}
	\renewcommand{\footrulewidth}{0.4pt}
	\renewcommand{\headrulewidth}{0.0pt}
}

% no gap between page line and footnote
%\setlength{\footskip}{17pt}
% https://tex.stackexchange.com/questions/164367/how-to-make-footnotes-appear-at-bottom-of-the-footers-bar

%% START FOOTNOTE FOOTNOTE BELOW HORIZONTAL LINE
\makeatletter
\renewcommand\footnoterule{%
	\kern20\p@
	\hrule\@width\linewidth
	\kern2.6\p@}
\makeatother

\fancypagestyle{phdfancyfootnote}{%
	
	\pagestyle{fancy}
	\fancyhead{} % Clean headers
	\fancyhead[LE,RO]{\THETITLE}
	\fancyhead[RE,LO]{\leftmark}
	\fancyfoot[C]{\thepage}
	\renewcommand{\footrulewidth}{0.0pt}
	\renewcommand{\headrulewidth}{0.4pt}
	\renewcommand{\chaptermark}[1]{\markboth{\thechapter. {\slshape{##1}}}{}} 
}

\fancypagestyle{phdfancyfootnotewithoutheader}{%
	
	\pagestyle{fancy}
	\fancyhead{} % Clean headers
	\fancyhead[LE,RO]{}
	\fancyhead[RE,LO]{}
	\fancyfoot[C]{\thepage}
	\renewcommand{\footrulewidth}{0.0pt}
	\renewcommand{\headrulewidth}{0.0pt}
	\renewcommand{\chaptermark}[1]{\markboth{\thechapter. {\slshape{##1}}}{}} 
}

%% END FOOTNOTE FOOTNOTE BELOW HORIZONTAL LINE

%% BETTER FOOTNOTE WITH FANCY STYLE
\newcommand\footnoteurl[2]{\thispagestyle{empty}\thispagestyle{phdfancyfootnote}\footnote{\href{#2}{#1}}}

\newcommand{\footnoteplain}[1]{\thispagestyle{empty}\thispagestyle{phdfancyfootnote}\footnote{#1}}

%% FOOTNOTE WITHOUT HEADER
\newcommand\footnoteurlwithoutheader[2]{\thispagestyle{empty}\thispagestyle{phdfancyfootnotewithoutheader}\footnote{\href{#2}{#1}}}

\newcommand\footnotewithoutheader[1]{\thispagestyle{empty}\thispagestyle{phdfancyfootnotewithoutheader}\footnote{#1}}

\newenvironment{startpagenumbering}[2][0] {

	%	\pagestyle{plain}
	
	\pagestyle{phdfancy}
	%	\setlength{\footskip}{17pt}
	\pagenumbering{arabic}
	\setcounter{page}{#1}{
		#2
	}

	% rework plain style to fancy
	\fancypagestyle{plain}{
		
		\pagestyle{fancy}
		\fancypagestyle{phdfancyplain}{%
			\fancyhead{} % Clean headers
			\fancyfoot[C]{\thepage}
			\renewcommand{\footrulewidth}{0.4pt}
			\renewcommand{\headrulewidth}{0.0pt}
		}
		
		\pagestyle{phdfancyplain}
	}

	%% REWORK TITLE CHAPTERS
	\definecolor{gray75}{gray}{0.75}
	\newcommand{\hsp}{\hspace{0pt}}
	\titleformat{\chapter}[hang]{\vspace{-4.8cm}\flushright
		\fontseries{b}\fontsize{80}{100}\selectfont}{\fontseries{b}\fontsize{100}{130}\selectfont \textcolor{gray75}\thechapter\hsp}{0pt}{\\ \vspace{-1cm}\Huge\bfseries}[\titlerule\vspace{-1cm}]

%\titleformat{\chapter}[display]{}{}{2ex}{
%	\begin{center}
%		\hspace*{\dimexpr0em-10pt\relax}
%		\advance\hsize4em\advance\hsize10pt\rule[0.5ex]{4em}{1pt}\hspace{10pt}
%		\begin{varwidth}{\textwidth}\Large\thechapter\end{varwidth}
%		\hspace*{10pt}\rule[0.5ex]{4em}{1pt}
%	\end{center}
%	\vskip 1cm
%	\begin{center}
%		\hspace*{\dimexpr5em-10pt\relax}
%		\Large #1
%	\end{center}
%}

}{}
